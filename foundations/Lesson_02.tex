\documentclass[12pt,twoside]{article}
\input{../sears.tex}
\fancyfoot[C]{\tiny \begin{tabular}[b]{c}
    Remixed from \textit{Prealgebra 2e} by Lynn Marecek, MaryAnne Anthony-Smith, Andrea Honeycutt Mathis. \\
    Access for free at https://openstax.org/books/prealgebra-2e/pages/1-introduction
\end{tabular}}


\title{Lesson 2}
\author{Foundations of College Algebra}
\date{}

\begin{document}

\maketitle

\thispagestyle{fancy}

\section*{Simplest Form of a Fraction}

\subsection*{Definition}
\begin{itemize}\setlength{\itemsep}{-\parsep}
\item \textbf{Equivalent fractions} are fractions that have the same value.
\item A fraction is considered \textbf{simplified} if there are not common factors in the numerator and denominator.
\end{itemize}

\subsection*{How To - Simplify a Fraction}
\begin{enumerate} \setlength{\itemsep}{-\parsep}
\item Rewrite the numerator and denominator to show he common factors. If needed, factor the numerator and denominator into prime factors.
\item Simplify, using the equivalent fractions property, by removing common factors.
\item Multiply any remaining factors.
\end{enumerate}

\subsection*{Examples}
\begin{multienumerate}
  \mitemxxx{$\frac{10}{15}$}{$\frac{12}{16}$}{$\frac{210}{385}$}
\end{multienumerate}

\subsection*{You Try}
\begin{multienumerate}
  \mitemxxx{$\frac{8}{12}$}{$\frac{40}{88}$}{$\frac{120}{252}$}
\end{multienumerate}\vspace\fill

\section*{Multiplying Fractions}

\subsection*{How To - Fraction Multiplication}
If $a$, $b$, $c$, and $d$ are numbers where $b \neq 0$ and $d \neq 0$, then $$\frac{a}{b} \cdot \frac{c}{d} = \frac{ac}{bd}.$$ \\
\textbf{Hint} It is easier to cancel factors before multiplying numerators and denominators.

\subsection*{Examples}
Multiply and write the answer in simplest form.
\begin{multienumerate}
\mitemxxx{$\frac59 \cdot \frac3{10}$}{$\frac{63}{84} \cdot \frac{44}{90}$}{$\frac{27}{32} \cdot \frac{10}{13} \cdot \frac{16}{30}$}
\end{multienumerate}

\subsection*{You Try}
Multiply and write the answer in simplest form.
\begin{multienumerate}
\mitemxxx{$\frac45 \cdot \frac27$}{$\frac38 \cdot \frac4{15}$}{$\frac{33}{60} \cdot \frac{40}{88}$}
\end{multienumerate} \vspace\fill

\pagebreak

\section*{Multiplying Fractions and Mixed Numbers}

\subsection*{How To - Multiply or Divide Mixed Numbers}
\begin{multicols}{3}
  \begin{enumerate} \setlength{\itemsep}{-\parsep}
  \item Convert the mixed numbers to improper fractions. \columnbreak
  \item Follow the rules for fraction multiplication or division. \columnbreak
  \item Simplify if possible.
  \end{enumerate}
\end{multicols}

\subsection*{Examples}
Multiply and write your answer in simplest form.
\begin{multienumerate}
  \mitemxxx{$3 \frac13 \cdot \frac58$}{$2 \frac 45 \cdot 1 \frac78$}{$4 \frac23 \cdot 1 \frac18$}
\end{multienumerate}

\subsection*{You Try}
Multiply and write your answer in simplest form.
\begin{multienumerate}
  \mitemxxx{$4 \frac38 \cdot \frac7{10}$}{$2 \frac 25 \cdot 2 \frac29$}{$4 \frac49 \cdot 5 \frac{13}{16}$}
\end{multienumerate} \vspace\fill

\section*{Solving Problems by Multiplying Fractions}

\subsection*{You Try}
A booth at the county fair sells fudge by the pound.
Their award winning ``Chocolate Overdose'' fudge contains $2 \frac23$ cups of chocolate per pound.
\begin{enumerate}
\item How many cups of chocolate chips are in a half-pound of fudge? \vspace\fill
\item The owners of the booth need to make the fudge in 10-pound batches. How many chocolate chips do they need to make a 10-pound batch?
  Write your results as an improper fraction and as a mixed number. \vspace\fill
\end{enumerate}

\section*{Finding Reciprocals of Fractions}

\subsection*{Definition}
The \textbf{reciprocal} of the fraction $\frac{a}{b}$ is $\frac{b}{a}$, where $a \neq 0$ and $b \neq 0$.
A number and its reciprocal have a product of $1$.

\subsection*{Examples}
Find the reciprocal of each number.
\begin{multienumerate}
\mitemxxx{$\frac49$}{$\frac1{11}$}{$13$}
\end{multienumerate} \vspace\fill

\pagebreak

\section*{Dividing Fractions}

\subsection*{How To - Divide Fractions}
To divide two fractions, multiply the first fraction by the reciprocal of the second.
If $a$, $b$, $c$, and $d$ are numbers where $b \neq 0$, $c \neq 0$, and $d \neq 0$, then
$$\frac{a}{b} \div \frac{c}{d} = \frac{a}{b} \cdot \frac{d}{c}.$$
\textbf{Hint} You should flip the second fraction before doing any other steps.

\subsection*{Examples}
Divide and write the answer in simplified form.
\begin{multienumerate}
\mitemxx{$\frac25 \div \frac37$}{$\frac7{18} \div \frac{14}{27}$}
\end{multienumerate}

\subsection*{You Try}
Divide and write the answer in simplified form.
\begin{multienumerate}
\mitemxx{$\frac7{27} \div \frac{35}{36}$}{$\frac5{14} \div \frac{15}{28}$}
\end{multienumerate} \vspace\fill

\section*{Dividing Fractions and Mixed Numbers}

\subsection*{Examples}
Divide and write your answer in simplified form.
\begin{multienumerate}
\mitemxx{$ 3 \frac47 \div 5$}{$18 \frac34 \div 3 \frac34$}
\end{multienumerate}

\subsection*{You Try}
Divide and write your answer in simplified form.
\begin{multienumerate}
\mitemxx{$ 5 \frac13 \div 4$}{$9 \frac35 \div 1 \frac35$}
\end{multienumerate} \vspace\fill

\section*{Solving Problems by Dividing Fractions}
\subsection*{You Try}
Traxel's Jewelry paid \$150 for a $\frac38$-carat gem.
At this price, what is the cost of one carat. \vspace\fill


\end{document}
