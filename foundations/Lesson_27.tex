\documentclass[12pt,twoside]{article}
\usepackage{amsmath}

\usepackage{times}
\usepackage{helvet}

%\usepackage{fontspec}
%\setmainfont{Liberation Serif}
%\setsansfont{Liberation Sans}

\usepackage{geometry}
\geometry{               
	letterpaper,
	bottom=0.5in,
	top=0.5in,
	inner=1in,
	outer=0.5in,
        footskip=4ex
}

\usepackage{graphicx}

\usepackage{titling}
\pretitle{\begin{center}\LARGE\sffamily}
\posttitle{\par\end{center}\vspace{-4ex}}
\preauthor{\begin{center}\large\sffamily}
\postauthor{\par\end{center}\vspace{-12ex}}
\setlength{\droptitle}{-60pt}



\usepackage{multienum}

\usepackage{titlesec}
\titleformat*{\section}{\Large\sffamily}
\titleformat*{\subsection}{\sffamily}

\titlespacing\section{2em}{0.5ex plus 0.2ex minus 0.1ex}{0pt}
\titlespacing\subsection{1em}{0.5ex plus 0.2ex minus 0.1ex}{0pt}

\usepackage{fancyhdr}
\pagestyle{fancy}

\fancyhf{}
\renewcommand{\headrule}{}
\fancyfoot[LE]{\thepage}
\fancyfoot[RO]{\thepage}

\usepackage{multicol}

\setlength{\parindent}{0pt}

\fancyfoot[C]{\tiny \begin{tabular}[b]{c}
    Remixed from \textit{Prealgebra 2e} by Lynn Marecek, MaryAnne Anthony-Smith, Andrea Honeycutt Mathis \\
    Access for free at https://openstax.org/books/prealgebra-2e/pages/1-introduction
\end{tabular}}


\title{Lesson 27}
\author{Foundations of College Algebra}
\date{}

\begin{document}

\maketitle

\thispagestyle{fancy}
\section*{Recognize Perfect Square Trinomials}

\subsection*{Perfect Square Trinomial Pattern}
$$a^2 + 2ab + b^2 = \left(a + b \right)^2\quad \text{and} \quad a^2 - 2ab + b^2 = \left(a-b \right)^2.$$

\subsection*{Examples}
Determine whether the following trinomials are perfect squares.
\begin{multienumerate}
\mitemxx{ $25v^2+10v+4$}{$49x^2 - 28xy + 4y^2$}
\end{multienumerate}

\subsection*{Your Turn}
Determine whether the following trinomials are perfect squares.
\begin{multienumerate}
\mitemxx{$16y^2 + 24y + 9$}{$100x^2-40x+1$}\vspace\fill
\end{multienumerate}

\section*{Factor Perfect Square Trinomials}
\subsection*{Examples}
Factor completely.
\begin{multienumerate}
\mitemxx{$36s^2 + 84s + 49$}{$25r^2-60rs+36s^2$}
\mitemx{$75u^3-30u^2v+3uv^2$}
\end{multienumerate}

\subsection*{Your Turn}
Factor completely.
\begin{multienumerate}
\mitemxx{$49s^2 + 154s + 121$}{$64z^2-16z+1$}\vspace\fill
\mitemxx{$10k^2+80k+160$}{$64x^2-96x+36$}\vspace\fill
\end{multienumerate}

\pagebreak

\section*{Factor the Difference of Two Squares}
\subsection*{Difference of Squares Pattern}
$$a^2 - b^2 = \left(a + b\right) \left(a - b\right)$$
\subsection*{Examples}
Factor each binomial completely.
\begin{multienumerate}
		\mitemxx{$x^2-16$}{$49x^2-81y^2$}
		\mitemx{$98r^3-72r$}
\end{multienumerate}
\subsection*{You Try}
Factor each binomial completely.
\begin{multienumerate}
		\mitemxx{$n^2-9$}{$25v^2 - 1$}\vspace\fill
		\mitemxx{$36p^2 - 49q^2$}{$5q^2 - 45$} \vspace\fill
\end{multienumerate}

\section*{Solve Quadratic Equations by Factoring}
\subsection*{The Zero Product Property}
If $a \cdot b = 0$ , then either $a=0$, $b=0$, or both.
\subsection*{Examples}
Solve the equations.
\begin{multienumerate}
		\mitemxx{$(x+1)(x-4) = 0$}{$(5n-2)(6n-1)=0$}
\end{multienumerate}
\subsection*{You Try}
Solve the equations.
\begin{multienumerate}
		\mitemxx{$(x-3)(x+7) = 0$}{$(3a-10)(2a-7) = 0$} \vspace\fill
\end{multienumerate}

\pagebreak

\subsection*{Definition}
The standard form of a quadratic equation is $ax^2 + bx + c = 0$.
Here, $a$, $b$, and $c$ are constants.

\subsection*{Procedure}
To solve a quadratic equation by factoring:

\begin{multicols}{2}
Step 1. Write the quadratic equation in standard form.

Step 2. Factor the quadratic expression. 
	\vfill\null \columnbreak

Step 3. Use the Zero Product Property to write two linear equations.

Step 4. Solve the linear equations.

Step 5: Check.
\end{multicols}

\subsection*{Examples}
Solve the equations.
\begin{multienumerate}
	\mitemxx{$y^2-8y+15=0$}{$2y^2=13y+45$}
	\mitemxx{$144q^2 = 25$}{$(3x-8)(x-1) = 3x$}
\end{multienumerate}

\subsection*{You Try}
Solve the equations.
\begin{multienumerate}
	\mitemxx{$x^2+7x+12=0$}{$n^2 = 5n-6$} \vspace\fill
	\mitemxx{$4b^2+7b=-3$}{$4m^2 = 17m-15$} \vspace\fill
	\mitemxx{$49m^2 = 144$}{$(y-3)(y+2) = 4y$} \vspace\fill
\end{multienumerate}

\end{document}
