\documentclass[12pt,twoside]{article}
\input{../sears.tex}
\fancyfoot[C]{\tiny \begin{tabular}[b]{c}
    Remixed from \textit{Prealgebra 2e} by Lynn Marecek, MaryAnne Anthony-Smith, Andrea Honeycutt Mathis. \\
    Access for free at https://openstax.org/books/prealgebra-2e/pages/1-introduction
\end{tabular}}


\title{Lesson 3}
\author{Foundations of College Algebra}
\date{}

\begin{document}

\maketitle

\thispagestyle{fancy}

\section*{Add Fractions with a Common Denominator}

\subsection*{How To - Fraction Addition}

If $a$, $b$, and $c$ are numbers where  $c \neq 0$, then
$$ \frac{a}{c} + \frac{b}{c} = \frac{a + b}{c}$$

To add fractions with a common denominator, add the numerators and place the sum over the common denominator.

\textbf{Note:} Subtracting fractions works the exact same way.

\subsection*{Examples}
Find each sum.
\begin{multienumerate}
  \mitemxxx{$\frac49 + \frac19$}{$\frac16 + \frac36$}{$\frac38 + \frac38$}
\end{multienumerate}

\subsection*{You Try}
Find each sum.
\begin{multienumerate}
  \mitemxxx{$\frac29 + \frac59$}{$\frac9{15} + \frac7{15}$}{$\frac3{16} + \frac7{16}$}
\end{multienumerate}
\vspace\fill

\section*{Subtract Fractions with a Common Denominator}

\subsection*{Examples}
\begin{multienumerate}
\mitemxx{$\frac{23}{24} - \frac{14}{24}$}{$\frac{19}{28} - \frac7{28}$}
\end{multienumerate}

\subsection*{You Try}
\begin{multienumerate}
\mitemxx{$\frac58 - \frac28$}{$\frac7{12} - \frac5{12}$}
\end{multienumerate}
\vspace\fill

\section*{Solve Problems by Adding or Subtracting Fractions with a Common Denominator}

\subsection*{You Try}
Trail Mix Jacob is mixing together nuts and raisins to make trail mix. He has $\frac6{10}$ of a pound of nuts and $\frac3{10}$ of a pound of raisins. How much trail mix can he make?

\vspace\fill

\pagebreak

\section*{Find the Least Common Multiple Using Lists}

\subsection*{Definitions}
\begin{itemize}\setlength{\itemsep}{-\parsep}
\item A number is a \textbf{multiple} of  $n$  if it is the product of a counting number and  $n$.
For example, the multiples of $4$ are:
$$4, 8, 12, 16, 20, \dots .$$
\item The smallest number that is a multiple of two numbers is called the \textbf{least common multiple (LCM)}.
\item The \textbf{least common denominator (LCD)} of two fractions is the least common multiple (LCM) of their denominators.
\end{itemize}

\subsection*{How To - Find the Least Common Multiple Using Lists}
%\begin{multicols}{2}
\begin{enumerate} \setlength{\itemsep}{-\parsep}
\item List the first several multiples of each number.
\item Look for multiples common to both lists. If there are no common multiples in the lists, write out additional multiples for each number.
\item Look for the smallest number that is common to both lists.
\item This number is the LCM.
\end{enumerate}
%\end{multicols}

\subsection*{Examples}
Find the least common multiple by listing multiples.
\begin{multienumerate}
\mitemxxx{$8, 12$}{$12, 16$}{$60, 75$}
\end{multienumerate}

\subsection*{Examples}
Find the least common multiple by listing multiples.
\begin{multienumerate}
\mitemxxx{$4, 3$}{$6, 15$}{$20, 30$}
\end{multienumerate}
\vspace\fill

\section*{Write Equivalent Fractions}

\subsection*{Fact}
If $a$, $b$, $c$ are whole numbers where $b \neq 0$, $c\neq0$,
then
$$ \frac{a}{b} = \frac{a\cdot c}{b\cdot c} \text{ and } \frac{a \cdot c}{b \cdot c} = \frac{a}{b} .$$

\subsection*{Examples}
Change to equivalent fractions with the LCD.
\begin{multienumerate}
\mitemxx{$\frac34$ and $\frac56$}{$\frac8{15}$ and $\frac{11}{24}$}
\end{multienumerate}

\subsection*{You Try}
Change to equivalent fractions with the LCD.
\begin{multienumerate}
\mitemxxx{$\frac13$ and $\frac14$}{$\frac5{12}$ and $\frac78$}{$\frac{13}{16}$ and $\frac{11}{12}$}
\end{multienumerate}
\vspace\fill

\pagebreak

\section*{Add Unlike Fractions}

\subsection*{How To - Add or Subtract Fractions with Different Denominators}
\begin{multicols}{2}
\begin{enumerate}\setlength{\itemsep}{-\parsep}
\item Find the LCD.
\item Convert each fraction to an equivalent form with the LCD as the denominator.
\item Add or subtract the fractions.
\item Write the result in simplified form.
\end{enumerate}
\end{multicols}

\subsection*{Examples}
Add.
\begin{multienumerate}
\mitemxxx{$\frac12 + \frac15$}{$\frac7{12} + \frac{11}{15}$}
{$\frac{39}{56} + \frac{22}{35}$}
\end{multienumerate}

\subsection*{You Try}
\begin{multienumerate}
\mitemxxx{$\frac13 + \frac15$}{$\frac5{12} + \frac38$}
{$\frac{9}{20} + \frac{17}{30}$}
\end{multienumerate}

\vspace\fill

\section*{Subtract Unlike Fractions}

\subsection*{Examples}
\begin{multienumerate}
\mitemxx{$\frac7{12} - \frac9{16}$}{$\frac{19}{24} - \frac7{15}$}
\end{multienumerate}


\subsection*{You Try}
\begin{multienumerate}
\mitemxx{$\frac7{16} - \frac5{12}$}{$\frac{11}{12} - \frac38$}
\end{multienumerate}

\vspace\fill

\section*{Solving Problems by Adding or Subtracting Fractions}

\subsection*{You Try}
Laronda is making covers for the throw pillows on her sofa. For each pillow cover, she needs  $\frac3{16}$  yard of print fabric and  $\frac38$  yard of solid fabric. What is the total amount of fabric Laronda needs for each pillow cover?

\vspace\fill

\end{document}
